\documentclass{article} % Define la clase del documento, en este caso, un artículo
\usepackage[letterpaper,margin=3cm]{geometry} % Configura el tamaño del papel y los márgenes del documento
\usepackage{graphicx} % Permite la inserción de imágenes
\usepackage[spanish]{babel}% Activar esta configuración para informes en español, ajusta el idioma del documento
\usepackage[usenames]{color} % Permite el uso de colores definidos por nombre en el documento
\usepackage{hyperref} % Habilita enlaces y referencias dentro del documento
\hypersetup{colorlinks=true, linkcolor = black, citecolor= black} % Configura el color de los enlaces y citas
\usepackage{booktabs} % Proporciona comandos para crear tablas de alta calidad
\usepackage{natbib} % Permite el uso de citas y referencias bibliográficas con diferentes estilos
\usepackage{tikz} % Permite la creación de gráficos y diagramas vectoriales directamente en LaTeX
\usepackage{float} % Para controlar la posición de los elementos flotantes, como imágenes, con la opción [H]
\usepackage{diagbox} % Permite crear celdas con líneas diagonales en tablas
\usepackage{listings} % Permite la inclusión y formateo de código fuente en el documento
\usepackage{xcolor} % Paquete para definir y usar colores en el documento
\usepackage{parskip} % Añade espacio entre párrafos en lugar de sangrías
\usepackage{fancyhdr} % Permite personalizar encabezados y pies de página
\usepackage{amsmath} % Proporciona una amplia variedad de entornos y comandos matemáticos

\pagestyle{fancy} % Usa el estilo fancyhdr
\fancyhf{} % Borra todos los encabezados y pies de página
\renewcommand{\headrulewidth}{0pt}
\renewcommand{\footrulewidth}{0pt} % Desactiva la línea horizontal predeterminada en el pie
\setlength{\headheight}{2cm} % Ajusta la altura del encabezado para hacer espacio para la línea
\fancyhead[L]{\raisebox{0.20cm}{\textbf{Métodos y Técnicas de Construcción}}} % Añade el texto en la parte izquierda del encabezado, subiéndolo ligeramente
\fancyhead[R]{\raisebox{0.1cm}{\includegraphics[width=0.25\linewidth]{LOGO_UNIVERSIDAD.jpg}}} % Añade la imagen en la parte derecha del encabezado y súbela un poco
\fancyhead[C]{\rule{\textwidth}{0.6pt}} % Añade una línea horizontal superior centrada
\fancyfoot[C]{\rule{\textwidth}{0.6pt}} % Añade una línea horizontal en el pie de página centrada
\fancyfoot[R]{\raisebox{-1.5\baselineskip}{\thepage}} % Coloca el número de página a la derecha, con suficiente espacio debajo de la línea
\geometry{top=3cm, bottom=2.5cm} % Ajusta los márgenes superior e inferior

% Definición de colores al estilo Visual Studio Code
\definecolor{codegreen}{rgb}{0.25,0.49,0.48} % Comentarios
\definecolor{codegray}{rgb}{0.5,0.5,0.5} % Números y anotaciones
\definecolor{codepurple}{rgb}{0.58,0,0.82} % Palabras clave
\definecolor{backcolour}{rgb}{0.95,0.95,0.92} % Color de fondo

% Configuración del estilo de las celdas de código
\lstset{
    backgroundcolor=\color{backcolour},   % color de fondo; necesita que el paquete color o xcolor esté cargado
    commentstyle=\color{codegreen},       % estilo de comentarios
    keywordstyle=\color{codepurple},      % estilo de palabras clave
    numberstyle=\tiny\color{codegray},    % estilo de los números de línea
    stringstyle=\color{red},              % estilo de las cadenas de texto
    basicstyle=\ttfamily\small,           % estilo del texto básico
    breakatwhitespace=false,              % ajustes de líneas sólo en espacios en blanco
    breaklines=true,                      % ajustar las líneas si son muy largas
    captionpos=b,                         % posición de la leyenda (abajo)
    keepspaces=true,                      % preserva los espacios en el texto; útil si se usa monoespaciado
    numbers=left,                         % dónde poner los números de línea
    numbersep=5pt,                        % qué tan lejos están los números de línea del código
    showspaces=false,                     % mostrar espacios con subrayados particulares; reemplaza 'showstringspaces'
    showstringspaces=false,               % subrayar los espacios dentro de las cadenas solo
    showtabs=false,                       % mostrar tabulaciones en el código con subrayados particulares
    tabsize=2,                            % tamaños de tabulación a 2 espacios
    language=TeX,                         % lenguaje del código
    morecomment=[l]\#,                    % reconocer # como inicio de comentario en Python
    frame=single,                         % agregar un marco simple alrededor del código
    rulecolor=\color{black}               % color del marco
}

\begin{document}
%----------------------------------------------------------------------------------------
%   PORTADA
%Modificar desde aqui en adelante
%----------------------------------------------------------------------------------------
\begin{titlepage}%Inicio de la carátula, solo modificar los datos necesarios
\newcommand{\HRule}{\rule{\linewidth}{0.5mm}} 
\center 
%----------------------------------------------------------------------------------------
%	ENCABEZADO
%----------------------------------------------------------------------------------------
\includegraphics[width=10cm]{LOGO_UNIVERSIDAD.jpg}\\ % Si esta plantilla se copio correctamente, va a llevar la imagen del logo de la facultad.OBS: Es necesario incluir el paquete: graphicx
\vspace{3cm}
%----------------------------------------------------------------------------------------
%	SECCION DEL TITULO
%----------------------------------------------------------------------------------------
\HRule \\[0.4cm]
{ \huge \bfseries Tarea 2}\\[0.4cm] % Titulo del documento
{ \huge \bfseries Métodos y Técnicas de Construcción}\\[0.4cm] % Titulo del documento
\HRule \\[1.5cm]
 \vspace{5cm}
%----------------------------------------------------------------------------------------
%	SECCION DEL AUTOR
%----------------------------------------------------------------------------------------
\begin{flushright}
    { \textbf{Profesor:}\\
    Jose Tramon\\
    \vspace{0.2cm}
    \textbf{Alumnos:}\\
    Bernardo Caprile Canala-Echevarría\\
    Pedro Tomás Valenzuela Bejares\\
    \vspace{0.2cm}

}
\end{flushright}
\vspace{1cm}
%----------------------------------------------------------------------------------------
%	SECCION DE LA FECHA
%----------------------------------------------------------------------------------------
{\large \textbf{\today}}\\[2cm] % El comando \today coloca la fecha del dia, y esto se actualiza con cada compilacion, en caso de querer tener una fecha estatica, reemplazar el \today por la fecha deseada
\end{titlepage}
%----------------------------------------------------------------------------------------
%  INDICE
%----------------------------------------------------------------------------------------
\newpage
\tableofcontents
\thispagestyle{plain} % Deshabilita el encabezado en la página del índice
\thispagestyle{empty} % Deshabilita el número de página en la página del índice
\newpage

%Se puede agregar un indice de figuras si es nesesario
%\newpage
%\listoffigures 
%\thispagestyle{plain} % Deshabilita el encabezado en la página del índice %
%\thispagestyle{empty}
%\newpage
%----------------------------------------------------------------------------------------
%   ACÁ EMPIEZA EL INFORME
\setcounter{page}{1} % Reinicia el contador de páginas
%-------
\section{Introducción}
Este informe presenta la programación, valorización y seguimiento de avance de la construcción de un edificio de bodega. El objetivo es planificar las actividades necesarias, calcular los costos y preparar los pagos parciales para la obra.

Primero, se realizó la programación de actividades en MS Project, siguiendo un orden constructivo lógico, con estimaciones de duración y relaciones de precedencia entre tareas. Además, se identifica la ruta crítica y se establece un seguimiento del avance en diferentes etapas del proyecto.

Luego, se valorizó cada partida con precios actualizados a enero de 2024, considerando costos directos, gastos generales y un porcentaje de utilidad. También se preparan estados de pago al 15\% y 35\% de avance, ajustados por el IPC.

Este informe busca ofrecer una visión clara y estructurada del proceso de planificación y valorización, asegurando un seguimiento efectivo del proyecto y control de los costos.

\newpage
\section{Programación de Actividades}

\begin{figure}[H]
    \centering
    \includegraphics[width=1\linewidth]{GRAFICOS/P1A.png}
    \caption{Programación de actividades en MS Project}
    \label{fig:programacion}
\end{figure}

\newpage
\section{Precios Unitarios}

A continuacion se presenta la tabla de precios unitarios de ciertos elementos utilizados en las partidas de la obra.

\begin{table}[H]
    \centering
    \caption{Tabla de Precios Unitarios}
    \label{tab:precios_unitarios}
    \begin{tabular}{lllll}
    \toprule
                                           Descripción & Unidad &  Cantidad & Precio Unitario & Precio Total \\
    \midrule
    \textbf{Relleno Estructural bajo 6" (entre Fundaciones ...)} &  &  & & \\
                            Estabilizado c/ Flete 15km &     m3 &       1.2 &           18600 &        22320 \\
                        Placas compatadoras de 2000 kg &    dia &      0.16 &            5598 &          896 \\
                                             Jornalero &    dia &       0.2 &           13775 &         2755 \\
                         \textbf{Terraplén bajo radier bodegas} &     &   &  &  \\
                  Material para confeccionar terraplen &     m3 &       1.3 &            1000 &         1300 \\
                                       Agua industrial &     m3 &      0.06 &             680 &           41 \\
                                    Camion agua aljibe &    hor &    0.0074 &           22566 &          167 \\
                                  Motoniveladora 200HP &    hor &    0.0125 &           40000 &          500 \\
                             Rodillo compactador 10TON &    hor &      0.01 &           25000 &          250 \\
                                     Camion tolva 15m3 &    hor &    0.1257 &           28257 &         3552 \\
                                      Excavadora 20TON &    hor &    0.0125 &           30000 &          375 \\
                         Capataz moviemiento de tierra &    dia &  0.003986 &           29600 &          118 \\
                          Jornalera 40 horas semaneles &    dia &     0.016 &           12300 &          197 \\
    \textbf{Hormigón G-04 Emplantillado Fundaciones Galpon...} &     &   &  &  \\
                             Hormigon G-04 (provision) &     m3 &      1.05 &           47645 &        50028 \\
                                       Servicio bombeo &     m3 &         1 &            8240 &         8240 \\
                        Concretero colocacion hormigon &    dia &      0.49 &           23227 &        11382 \\
                                   Jornalero capachero &    dia &      0.02 &           19296 &          386 \\
                                   \textbf{Solera Tipo} A &     &   &  &  \\
                              Solera tipo A 90x16x30cm &     mt &         1 &            7250 &         7250 \\
                 Hormigon H-20 estructuras (provision) &     m3 &      0.04 &           53702 &         2149 \\
                            Sika antisol bidon 4.5 lts &    bid &         1 &            9490 &         9490 \\
                                              Perdidas &      \% &         5 &               - &          582 \\
                                               Albañil &    dia &     0.042 &           28500 &         1197 \\
                                             Jornalero &    dia &     0.042 &           13775 &          579 \\
    \bottomrule
    \end{tabular}
\end{table}

De esta tabla se puede observar que para todos los materiales se xonsidero el costo de la mano de obra y el costo de los materiales que se utilizan para la construcción de la obra.Esta información se obtuvo del Portal de Construcción ONDAC, donde se muestra todo lo necesasrio para poder llevar a cabo una unidad del elemento deseado.

Dentro de los criterios utilizados, se debe saber el tipo de actividad, quien la va a llevar a cabo y los materiales y herramientas que se utilizaran, es decir se deben cosniderar los recursos humanos y materiales que se utilizaran en la obra. Además se deben cosniderar los gastos generales, como lo son los trasportes y maquinaria utilizada. Finalmente algo a tener en consideracion en ciertos materiales es la perdida que puede ocurrir en el proceso de construcción, esta suele ser de un 5\% del total de material utilizado.

\section{Estado de Pago}

El estado de pago es un documento que se utiliza para llevar un control de los pagos que se deben realizar en una obra, en este caso se presentan los estados de pago al 15\% y 35\% de avance de la obra. Esto es un pago que se realiza a los contratistas por el trabajo realizado en la obra, donde se ve un avance financiero de la obra, asi siguiendo los gastos y pagos de la obra. Esta se paga en base a ciertas partidas especificas que representan el avance en cierto porcentaje de la obra. Para esto se calculan los precios totales de la partida, incluyendo los gastos generales y utilidad. Luego se calcula el porcentaje de avance de la obra y se calcula el monto a pagar en base a este porcentaje.

\end{document}